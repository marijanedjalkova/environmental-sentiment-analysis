
\chapter{User Manual}
\label{app:manual}

\initial{T}his project could be compiled and run in a number of different ways, both as a whole and part by part. In order to compile and run the project, certain dependencies and libraries have to be installed, such as \texttt{NLTK}, \texttt{preprocessor}, and \texttt{scikit-learn}. The project uses Python 2.7. All the necessary libraries are compiled in the \texttt{resources/requirements.txt} file of the project and are also attached in an appendix \ref{app:pyreqs} of this report. 

A lot of modules of the project could be run separately and can be run from the command line. To explore the functionality of each module, one could modify the \texttt{main()} methos for each of them. Bu default, most of them currently perform simple test of a module by giving it a simple task.

\par For example, to run and test the TextBlob classifier, run

\begin{center}
\texttt{python textblob\_classifier.py}
\end{center}

This will print the accuracy of the classifier, using the training data from the provided data set.

\par To run the rule-based classifier, run

\begin{center}
\texttt{python rb\_environmental\_classifier.py}
\end{center}

This will create a classifier and print classification outputs for a number of given sentences from the \texttt{main()} method.

\par
Running
\begin{center}
\texttt{python elections.py}
\end{center}

will create a Topic Model and perform ten-fold validation on it. The method will print averages for both accuracy and recall values obtained during the validation.

\par To test the Sentiment Model, \texttt{ngram\_classifier.py} has to be run. The command also accepts five arguments: 
\begin{enumerate}
    \item \textbf{Classifier name}: 
    \begin{itemize}
    \item ``Naive Bayes''
    \item ``MaxEntClassifier''
    \item ``SVM''
    \item ``DecisionTree''
  \end{itemize}
    \item \textbf{Number of tokens in an n-gram}: 1 produces the best model, 2 performs reasonably well, any other positive integer will create a model which is not likely to perform well. Any other number will not be validated.
    \item \textbf{Size of training data}: the dataset contains more than a million and a half of tweets. It has been found out empirically that a model does not need more than 300 000 tweets to train on in case of SVM. In a case of another classifier, this number will be even less. 
    \item \textbf{Size of testing data}: any non-negative integer that, summed with the previous parameter, will not give a sum larger than the size of the entire data set.
    \item \textbf{Name of the feature extractor}:
    \begin{itemize}
    \item \textbf{``ngram\_extractor''}: Simply extracts n-grams from the tweets and performs basic pre-processing.
    \item \textbf{``preprocessing\_extractor''}: Parses the tweet using the \texttt{preprocessor} library, then extracts n-grams. The most advanced feature extractor. 
    \item \textbf{``noun\_phrase\_extractor''}: Extracts noun phrases using external libraries. The slowest extractor, and the results are not satisfying.
  \end{itemize}
\end{enumerate}.

So, a command
\begin{center}
\texttt{python ngram\_classifier.py SVM 1 10000 200 preprocessing\_extractor}
\end{center}

will create and test an SVM for sentiment analysis from unigrams extracted using\\ \texttt{preprocessing\_extractor} from a training set of a size 10 000 and will test the model on 200 tweets.

The \texttt{twitter\_tool} module contains methods that show the implementation of the entire model. 

