\let\textcircled=\pgftextcircled
\chapter{Conclusions}
\label{chap:conclusions}

\par The project could be continued and a lot more Natural Language techniques could be applied to the models. The main constraint of this project has been time, and a lot of extremely interesting topics have only been touched slightly, and a lot more time and other resources could be spent on this project. For example, noun phrase extraction has been tried as a part of the sentiment analysis, and it would be very useful to have it implemented. However, in the scope of this project only allowed a library (NLTK) method to be used for noun phrase extraction, and it proved not to be very successful - each noun extraction would take almost 3 seconds, and a lot of noun phrases would not be detected. It wold be very interesting to implement a fast and light noun phrase extractor, however, it would require a lot more time. 

The project leaves a lot of open questions that have not been answered by the most recent research. For example, the Bag-of-Words is currently one of the most popular structures used in NLP. This is the simplest approach that allows to know almost nothing about the data that is being used. However, it would be interesting to see how the models that use the existing features of the data would perform. For example, is there a correlation between the location of the user and the sentiment of their posts? How can a tool detect idioms efficiently? 

Another unanswered question of NLP is whether irony and sarcasm could ever be detected by a machine, especially when people are not that good at detecting them themselves. 

Also, in the future work, it would be interesting to compare models created from the entire words comparing to models created from only stems of the words and also maybe models created purely from affixes of the data. This research would show how much of emotional information different morphemes hold.

Going on a level higher, it would be interesting to understand how much of emotional information different parts of speech hold. Intuitively, most of the information about the sentiment is being passed by adjectives and adverbs. Is it then possible to predict the sentiment of a sentence correctly using only words of these two classes? Would a model that contains no nouns perform much worse, and, if yes, how much worse?

If this project is ever continued, the person working on it should take all the points above into consideration. Despite the fact that the direction of the project had to be changed slightly, I hope that, with all the effort that has been put into this project, it will prove useful and contribute to a very interesting project and, hopefully, still move the Falkirk Council's project closer to the goal. 

It was incredibly interesting to work on this project, I learnt an immense amount of information, realised what the most interesting field of Computer Science is (for me, at least), received a lot of experience in Natural Language Processing, Machine Learning, working with APIs and so much more, and, generally, had an amazing time working on it. I also consider everything I have learnt throughout working on this project to be my largest achievement. 