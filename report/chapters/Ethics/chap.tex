\let\textcircled=\pgftextcircled
\chapter{Ethics}
\label{chap:ethics}

\initial{T}he project deals with the social media data which is publicly available and can be requested using the Twitter API. Research that uses public social media posts is becoming increasingly popular due to the availability of the data and the ease with which it can be obtained. Especially in Natural Language Processing, where a lot of user-created data is necessary to train different models, the Twitter platform is proving to be especially useful. 

Taking into account that using social media platforms for research is a relatively new phenomena, there are no clear rules yet defined for using this data, especially when it comes to the ethical reasoning. 

The main problem surrounding the usage of social network data is the fact that it is not clearly defined what constitutes private and public data. From the point of view of the availability of the data, the user defines what posts are public and what data is private using the settings usually available for them. This does not, however, define the cases in which the data could be used by third parties, such as advertisement companies and researchers. To partially solve this problem, some social media providers notify the users in the terms and conditions that their data could be used by third parties. Despite the fact that this mostly solves the problem from the legal point of view, the ethical issue remains. 

Traditionally, when an experimental study is conducted, all the participants are volunteers and all of them have to give consent for researchers to use their data in the study. They get explained the goals of the study and what they need to do and can ask any questions they have. Even though these rules impose certain constraints on some of the research (for example, scenarios where the scientist needs the participant not to know what is going to happen), they ensure that the study in conducted in an ethical manner. Another key concern for the researchers int he participants' anonymity. Traditionally, all the data used in research is anonymised. 

When the point of the study is to gather as much data as possible (from thousands of participants, probably, which is normally the case when working with Big Data), it is very hard to gather that many volunteers and receive consent to use their data from each one of them efficiently. When all of the necessary data is available online, and the users have given their consent for the data to be used by third parties by agreeing to the Terms and Conditions of the platform they are using, it can seem that the problem is solved and that one only needs to write a short script to obtain the data and use it as one pleases. 
Additionally, it is harder to anonymise the social media data due to the fact that most of the social media platforms allow searching for the public posts via keywords. So, for example, an author of a tweet could easily be identified using the keyword search (assuming, of course, that the tweet is not a re-tweet of a very popular tweet). The potential breach of anonymity increases a risk of harm of the participants, and it is impossible to know or predict which data is safe to use to avoid doing harm. Some studies suggest that when working with Big Data, the assumption that no other harm can be done to posts which have already been published is invalid \cite{humansubjectsbigdata}. 

When talking about environmental issues, research often looks into crisis data. Reuters reported about the record number of tweets after the Sandy hurricane struck in 2012 \cite{reuterssandy}, a number of studies perform have performed their studies based on the social media data received after different catastrophes such as earthquakes \cite{earthquakes} and other natural of man-made disasters \cite{haiti}.
However, a discussion has started about the fact that despite the fact that the data is available and the purpose of such studies is to help the humanity recover from the disasters faster, the use of this data might not be ethical \cite{crisisdata}. 

At this point, a lot of Big Data research is backed by different committees which make sure that the data is used ethically. The decision is normally based on several rules and the decision depends on how the data is used, whether the data is sensitive, whether it is allowed to collect the data and so on. For example, Glasgow University provides an ethics guide to researchers who want to work with Big Data. The guide was created at the University of Aberdeen and gives an ethics framework for those working with social media data. The framework asks the researchers questions such as: ``Can the social media user reasonably expect to be observed by strangers?", ``Are the research participants vulnerable? (i.e. children or vulnerable adults)", and ``Is the subject matter sensitive?" \cite{ethicsguide}. 
This project, too, works with social media data. Judging by this framework, the work done in this project is safe for the Twitter users. Twitter API allows data collection and users agree for their data to be used by agreeing to the Terms and Conditions. All the data for this project is anonymised and will not be published. In addition, despite the fact that the environment could be considered a sensitive topic, the existing data on Twitter does not mention any crisis, only routine environmental issues, so the project is not working with crisis data. In addition, ethical self-evaluation which is a mandatory part of writing a project at the University of St Andrews, also revealed no ethical issues. 

